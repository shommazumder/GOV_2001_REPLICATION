In conclusion, we do not refute the authors' argument that the effects of historical conflict persist today. We argue instead that the mechanisms connecting pre-colonial conflict may be more nuanced than \cite{BesleyRQ2014} suggest, and we have applied a series of techniques to both evaluate the strength of their findings and assess the comparative strength of their suggested mechanisms of persistence. More broadly, we believe that such an approach, combining modern empirical methods to identify and quantify the possible influence of different causal paths with a re-examination of their data, enables us to extract information on potential mechanisms the authors left unexplored while opening up new questions for future research. 

Firstly, by employing the LASSO algorithm and testing which subset of covariates best explain variation, we demonstrate that the incidence of pre-colonial conflict remains an important source of variation in modern conflict - even in spite of overdetermination and the coarse nature of their cross-country data. We again use this technique when analyzing the sub-national results to show which covariates, when interacted with historical conflict, explain the most variation in present-day social attitudes. Our findings build off of past research suggesting that that presence of the slave trade, interacted with historical conflict, is a consistent predictor of lower levels of inter-group trust and ethnic (rather than national) identification, and that other interactions with institutional variables seem to pick up important variation.

Secondly, building off the evidence of key interactive effects, we analyze the authors' causal mechanisms while excluding any post-treatment covariates. We find that the negative impact on trust and national identity and positive impact on ethnic identity still hold, but that the findings disappear when trying to predict subnational economic development. This suggests that historical conflict may not shape development in the traditional ``conflict begets conflict'' manner, but may be more nuanced through its effect on individuals' perceptions and identity. As such, this provides evidence that the causal channels connecting social attitudes with historical conflict appear distinct from those connecting modern economic outcomes. 

Our third section confirms that this mechanism may not be as clear as the authors suggest, and offers a path for future research. When pairing \cite{BesleyRQ2014} data with the original Afrobarometer survey data, we find that individuals in geographic areas with high levels of historical conflict not only have low levels of inter group trust, but also trust very few individuals and institutions, including members of their own ethnic group. More research should be done to better understand the complexities between historical conflict's effect on trust and identity formation, and why the presence of historical conflict may prompt some to ``identify'' with their ethnic group but not trust them any more than those in the outgroup. This could also help disentangle historical conflict's effect on modern economic outcomes, and if variation in trust levels could help to explain variation in present-day economic development.