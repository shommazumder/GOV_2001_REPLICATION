How does pre-colonial conflict shape contemporary economic and political development? Moreover, what are the mechanisms of persistence? In a recent paper, \citet{BesleyRQ2014} explore these questions in the context of Africa by assembling and analyzing a new national and subnational dataset of pre-colonial conflict in Africa. While much of the existing scholarship on African political economy points to the pernicious role of colonial institutions and practices in perpetuating economic and political underdevelopment, \citet{BesleyRQ2014} provide evidence that some of these roots of underdevelopment may actually lie long before the era of colonization \citep{nunn2008long,NunnWantchekon2011,MP2016}.

\citet{BesleyRQ2014} argue that pre-colonial conflict should shape modern development outcomes through their subsequent impact on economic, political and social outcomes. To explore these relationships, they provide three key sets of results. First, they demonstrate that, across a cross-section of countries, the number of wars in that country between 1400 and 1700 is strongly associated with the incidence of civil war in the post-independence era in Africa. Second, they show that individuals in areas that experiences pre-colonial conflict have significantly different social attitudes, as measured by lower levels of inter-group trust and higher levels of ethnic identification. Third, they disaggregate their results by focusing on war and economic outcomes at the grid cell level, and show that local economic activity is significantly lower, and the incidence of wars higher, in cells which experience historical conflict. 

Some of their models show clear signs of model dependence - for example, the significance of their cross-country results disappears when we simply turn the pre-colonial conflict variable into a dummy rather than continuous variable, and the result is also contingent on two extremely influential observations (as the authors themselves note on p.326). However, rather than demonstrating the fragility present in some of their results, we are more interested in building on the work done by the authors and, in particular, trying to parse the mechanisms which are just broadly sketched by the authors in terms of the persistent impact of pre-colonial conflict in Africa. 

Though \citet{BesleyRQ2014} provide an important substantive contribution to the literatures on African political economy, the persistence of historical events, and conflict, we argue that there are several avenues through which their piece can be improved to further enrich our understanding of the relationship between historical conflict and contemporary economic and political development. First, we argue that the authors of the original article are susceptible to the problem of overadjustment for covariates \citep{SchistermanColePlatt2009} in their first set of cross-country regression results. We implement variable selection methods using the LASSO algorithm \citep{tibshirani1996regression} in order to assess whether addressing the overadjustment issues modifies their first finding. We find that, in spite of the problems with their analysis, that their key explanatory variable is explaining important parts of the variation in modern day conflict outcomes.

Second, we maintain that the authors fail to utilize all of the information that they have in their data by assuming that historical conflict has constant effects across all units - an assumption that is generally implausible in most studies. Using their more disaggregated results, we explore automated approaches to selecting the most important heterogeneities left unexplored by their analysis. This approach suggests that certain interactions with institutional variables are especially important in accounting for differences in modern social attitudes. 

Third, building off the mechanisms that these heterogeneities suggest, we argue that when the authors do investigate their mechanisms, their strategy of controlling for post-treatment covariates opens their estimates up to unexpected biases. Accordingly, we estimate the average controlled direct effect of the treatment variable (historical conflict) holding post-treatment mediators (such as economic development) constant. This is done using an implementation of Sequential-G estimation as discussed in \citet{AcharyaBlackwellSen2016} and developed by \citet{JoffeGreene2009} and \citet{Vansteelandt2009}. From this, we find evidence that the causal pathways connecting historical conflict to social attitudes may be distinct from those connecting it to modern economic outcomes. 

Fourth and finally, we posit that the authors should have also utilized further data in their individual-level survey results that speak directly to their proposed mechanisms of persistence. In particular, a re-analysis of the Afrobarometer survey data suggests that the implications of pre-colonial conflict on trust attitudes are less clean-cut around ethnic and national lines than the authors suggest. In so doing, we provide empirical evidence on the existence and strength of different historical mechanisms connecting pre-colonial conflict with modern outcomes, with re-examined results which open up new questions for future research. 